\chapter*[Conclusão]{Conclusão}
\addcontentsline{toc}{chapter}{Conclusão}

A automatização do sistema de gerenciamento de energia da Faculdade Gama da Universidade de Brasília é uma iniciativa que irá otimizar o consumo e controle da energia elétrica utilizada no campus. A instalações de sensores de temperatura, presença e transmissão, a utilização de medidores inteligentes de energia e a implementação de um software de gerenciamento são processos necessários para que o projeto tenha sucesso.

As fontes de energia alternativas desempenham papel fundamental no processo, pois aproveitam características do local, como boa incidência solar e a geração de biomassa devido ao Restaurante Universitário para gerar eletricidade e contribuir com o sistema geral a ser implementado. Tais processos também são automatizados e são controlados e monitorados pelo sistema de gerenciamento do sistema. 

Portanto, um sistema ``SmartGrid'' na FGA integrado com fontes alternativas de energia é possível tecnicamente e pode solucionar as problemáticas levantadas no relatório do Ponto de Controle 1. A viabilidade econômica será analisada na próxima etapa, e assim poderá ser concluído se o projeto é realmente viável de ser implementado.