\part{Conclusão}
\chapter[Conclusão]{Conclusão}
% \addcontentsline{toc}{chapter}{Conclusão}
% \section{Conclusão}
A instalação de um sistema inteligente de gerenciamento de energia elétrica da Faculdade Gama da Universidade de Brasília veio da necessidade de reduzir os gastos com impostos relacionados a Concessionária de Energia de Brasília e diminuir a dependência da instituição com a CEB. Sendo assim, foram selecionadas duas fontes de energia alternativas, fotovoltaica e biogás, que aliadas aos componentes eletrônicos inteligentes, garantiram a elaboração de um projeto para melhorar a eficiência energética da FGA.

O desenvolvimento das pesquisas para a implementação do biodigestor na FGA mostraram que no campo da discussão das ideias, o projeto apresenta uma viabilidade mais animadora devido à disponibilidade geral de matéria orgânica para ser utilizada no biodigestor. Inicialmente o objetivo era, além da produção de energia elétrica, aliviar a disposição do lixo orgânico da cidade do Gama, porém, devido a questões burocráticas, desviou-se o curso da obtenção da matéria orgânica apenas para o Restaurante Universitário do Campus Gama da FGA.

Além das pesquisas de implementação do biodigestor, o projeto apresentou pesquisas relacionadas a implementação de  placas fotovoltaicas ao longo dos prédios da FGA para auxiliar na geração de energia total do campus ao longo de todo o mês. No total, seriam 620 placas fotovoltaicas policristalinas distribuidas nos telhados dos 3 prédios (Restaurante Universitário, UED e UAC) da FGA, onde o custo com compra dos componentes ficou em torno de R\$ 615.000,00 isso sem considerar os custos com instalação e possíveis custos com manutenção.

Para integrar as fontes de energia alternativas à rede elétrica do campus e garantir a automatização de todo o sistema de produção e gerenciamento de energia foram utilizados medidores inteligentes, sistema de transmissão dados e sensores de presença e luminosidade conectados a um software SCADA. Essas soluções permitem uma análise técnica e eficiente dos resultados energéticos por meio de gráficos dinâmicos em tempo real e relatórios práticos disponíveis em quaisquer plataformas, seja desktop ou mobile. 

Portanto, por meio dos cálculos de orçamento realizados sobre o projeto, concluímos que a viabilidade é prejudicada devido à baixa produção de energia. Fato que seria abrandado se fosse possível a captação de lixo orgânico na cidade do Gama para a utilização no biodigestor, além de uma maior área de placas solares na fotovoltaica. 
